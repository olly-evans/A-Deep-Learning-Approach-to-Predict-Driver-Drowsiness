\documentclass[a4paper, 11pt, twocolumn]{article}
\usepackage[utf8]{inputenc}
\usepackage{graphicx}
\usepackage{amsmath}
\usepackage{hyperref}
\usepackage{geometry}
\usepackage[numbers]{natbib}
\bibliographystyle{unsrtnat}
\geometry{a4paper, margin=25mm}


% Title and author details with clean, professional formatting
\title{
    \textit{\large Oliver James Evans} \\ % Subtitle in italics, smaller font    
    \textbf{\LARGE A Deep Learning Approach to Predict Driver Drowsiness} % Main title, slightly larger
}
\author{
    \normalsize ES327: Individual Project \\ % Module code, in normal size
    \normalsize Supervisor: Dr. Thomas Popham % Supervisor details
}
\date{
    \normalsize September 2024 % Submission date
}

\begin{document}
\maketitle

% ------------------------------------------------
% 1. Introduction
% ------------------------------------------------
\section{Introduction}
\label{sec:introduction}
The introduction should define the aim, objectives, scope, and stretch targets of your project. You should also justify the project by discussing its importance, the existing knowledge
 in the area, and the measurable benefits. Refer to key literature sources here.
\subsection{Aim and Objectives}
Clearly state the project's aim. Outline the specific objectives that will be used to achieve the aim.

\subsection{Scope}
Define the project’s scope, including the boundaries of what will and will not be covered.

\subsection{Justification and Benefits}
Discuss why this project is important, what problems it will solve, and what measurable benefits it will provide.

% ------------------------------------------------
% 2. Risk and Constraints
% ------------------------------------------------
\section{Risk and Constraints}
\label{sec:risks}
In this section, outline the key risks and constraints related to the project. Focus on those that are particularly relevant, and discuss how you plan to manage them.

\subsection{General Project Risks}
Discuss general risks such as time, uncertainty, and access to information.

\subsection{Specific Constraints}
Address specific constraints, such as:
\begin{itemize}
    \item \textbf{Time management}: How will you ensure timely progress?
    \item \textbf{Data security}: Managing data and intellectual property concerns.
    \item \textbf{Sustainability and environmental considerations}.
    \item \textbf{Health and safety}: Risks and mitigations.
    \item \textbf{Standards and legislation}: Which industry standards or legal requirements are relevant?
\end{itemize}

% ------------------------------------------------
% 3. Project Plan
% ------------------------------------------------
\section{Project Plan}
\label{sec:plan}
Provide an initial project plan, including milestones, timelines, and the criteria for measuring progress. Discuss how you will monitor and adjust the plan throughout the project lifecycle.

\subsection{Milestones}
List the key milestones and their respective deadlines.

\subsection{Progress Measurement}
Explain how progress will be tracked (e.g., logbooks, regular reviews).

\subsection{Risk Management}
Outline backup plans for unforeseen events, such as delays or data issues.

% ------------------------------------------------
% 4. Ethical Considerations
% ------------------------------------------------
\section{Ethical Considerations}
\label{sec:ethics}
Discuss the ethical implications of your project. Include whether an ethical approval is required, and if so, the steps taken to secure it. Confirm that the ethical review flowchart has been completed and the outcomes have been agreed upon with your supervisor.
% ------------------------------------------------
% 5. Initial Literature Review
% ------------------------------------------------
\section{Initial Literature Review}
\label{sec:litreview}
With any deep learning project the identification of a good available dataset is of upmost importance.\,This literature review will evaluate the available datasets related to driver
drowsiness, look at possible CNN architectures suitable to the task and assess the challenges associated with writing them.\, Deep learning and neural networks, playing a key role in
 the computer vision field have garnered substantial attention over the past years. UNFINISHED INTRO

\subsection{What is a Convolutional Neural Network?}
To understand a convolutional neural network we first need to make sense of a basic vanilla neural network. Neural networks are information processing paradigms that teach computers to
 process data in a way inspired by the human brain \citep{abiodun2018state}. They are built from basic units called neurons by analogy, which are interlinked by a number of weighted connections. The network
  learns through the modification of weights and biases of these various neurons, which are arranged in layers \citep{abdi1999neural}. 





% ------------------------------------------------
% 6. Feasibility of Implementation
% ------------------------------------------------
\section{Feasibility of Implementation}
\label{sec:feasibility}
Analyze the feasibility of implementing your project based on your initial findings. Use evidence and numerical justification where possible. Assess the challenges and whether they can be realistically addressed within the available time and resources.

% ------------------------------------------------
% 7. Conclusion
% ------------------------------------------------
\section{Conclusion}
\label{sec:conclusion}
Provide a conclusion that summarizes the feasibility of your project and reflects on the potential impact of the project if successfully implemented.

% ------------------------------------------------
% References
% ------------------------------------------------
\onecolumn
\bibliography{references} % This points to the references.bib file


\end{document}
